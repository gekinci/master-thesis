\chapter{Introduction}
A cooperative multi-agent system consists of relatively simple individuals, or agents, which comes together to solve tasks in a way that will benefit the population. In multi-agent systems, the well-being of the population strongly benefits from the cooperation between the agents. As limited abilities as the individuals might have, they give rise to a complex behaviour of the population through cooperation. These complex behaviors might be crucial for the survival of the population, e.g. running away from predators, rationing of vital substances. Such systems can be found in nature at cellular level \cite{Perkins2009a}. \\
Cells exist in stochastic environments. In order to maintain life, they are required to process noisy and fluctuating information coming from environment and response accordingly. In addition to the extracellular environment, the internal dynamics of cells are also found to be stochastic, such as their gene expressions \cite{Shahrezaei2008}. \\
Perkins \textit{et al.} \cite{Perkins2009a} argue that, as the extracellular environment is a stochastic process, the intracellular processing of these stochastic signals and choosing an appropriate response can only be probabilistic and consists of three main steps. First step is to infer from the noisy signals and make a prediction of the current or future state. Second is to make a proper choice of action considering the advantages and disadvantages, given the predictions. Final step is to take this action in a way that will contribute to the survival of the cell population. \\
Many studies took probabilistic approaches to explain the behaviour of cellular networks quantitatively and statistical inference is presented as a possible framework to explain the mechanisms that a cell may use to interpret the state of the environment from noisy signals. Libby \textit{et al} \cite{Libby2007} used Bayesian inference approach to model the gene expression of a bacterium in an environment with high and low level of sugar, and showed this model is consistent with the measurements. This approach is later extended for the situations where the signal fluctuates over time, e.g. non steady-state sugar concentration. Andrews \textit{et al} \cite{Andrews2016} proposed that the cell adopts to such environment by updating its beliefs in real-time. They modelled this decision-making mechanism as a sequential application of Bayesian inference, where the posterior probability that is inferred in the current step is used as a prior probability in the next step.\\
Bosher \textit{et al.} \cite{Bowsher2014} took a similar approach to explain cellular decision making, focusing on information theory. They argued that mutual information between the signal and the response is shown to be a suitable measure to quantify the cell's ability to infer. It is argued that if the mutual information between the signal to be inferred and the output of a signal transduction mechanism is high, only then the cell could be able to perform a high quality inference. For example, Bialek \textit{et al}, in their study of development of fruit fly, showed that the mutual information between gap gene expression and the position of nucleus is the information needed for each nucleus to determine their position in a cell. \\
The cooperation of the agents is optimized through the exchange of information between the agents. In honeybee and ant colonies, individuals share information about potential nest sites with the colony through recruitment signals, which are accumulated to acquire an intelligent collective decision on best nest site \cite{Franks2002}. In bacteria, individuals communicate by quorum sensing where they sense and produce hormone-like molecules called autoinducers. This process enables bacteria to observe the environment for the changes in the population or the presence of other bacteria and respond accordingly \cite{doi:10.1146/annurev.cellbio.21.012704.131001}.\\
The influence of communication on cooperative behavior is an area of ongoing research. PaarPorn \textit{et al.} \cite{Paarporn2018a} studies the optimal behaviour in a two-player cooperative game in a stochastic environment with and without information sharing under noisy channels. They find that majority logic strategy is optimal when the agents have intermediate reliable information from the environment and the information sharing is highly reliable. \\
Foerster \textit{et al.} \cite{Foerster2016}

\section{Motivation}

I wanna know how they communicate.


\section{Structure of the Thesis}
The remainder of the thesis is structured as follows.

\textbf{\cref{chap:2}} presents the theoretical background of this work. It reviews the details of continuous-time Bayesian networks and partially observable Markov decision processes, and introduces the sampling algorithms used in this work.

\textbf{\cref{chap:3}} is dedicated to the details of the problem. It explains how the frameworks are utilised, and presents the algorithms used for data generation and inference.

\textbf{\cref{chap:4}} presents the experimental results of the simulation and the inference.

\textbf{\cref{chap:5}} discusses the results, highlights the conclusions and reviews the limitations.

\textbf{\cref{chap:6}} gives suggestions for future directions and extentions to the research.