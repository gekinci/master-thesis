\chapter{Introduction}
A cooperative multi-agent system consists of relatively simple individuals, or agents, which come together to solve tasks in a way that will benefit the population. In such systems, the well-being of the population strongly benefits from the cooperation between the agents. The individuals might have limited abilities, however, they give rise to complex behaviour of the population through cooperation. These complex behaviours might be crucial for the survival of the population, e.g. running away from predators, rationing of vital substances, etc. Such systems can be found in nature even at the cellular level of living organisms \cite{Perkins2009a} among many other examples, such as swarms \cite{tan2013research}. \par
Cells exist in stochastic environments. To maintain life, they are required to process noisy and fluctuating information coming from the environment and respond accordingly. In addition to the extracellular environment, the internal dynamics of cells are also found to be stochastic, such as their gene expressions \cite{Shahrezaei2008}. Perkins and Swain (2009) \cite{Perkins2009a} argue that, as the extracellular environment is a stochastic process, the intracellular processing of these stochastic signals and the choice of an appropriate response can only be probabilistic. This process consists of three main steps. The first step is to infer the current or a future state from the noisy signals. Then the cell must choose appropriate actions considering advantages and disadvantages conditioned on its inference. The final step is to take this action in a way that will contribute to the survival of the cell population. \par
Many studies take probabilistic approaches to explain the behaviour of cellular networks quantitatively. Statistical inference is presented as a possible framework to explain the mechanisms that a cell may use to interpret the state of the environment from noisy signals. Libby \textit{et al.} (2007) \cite{Libby2007} use Bayesian inference approach to model the gene expression of a bacterium in an environment with high and low levels of sugar, and show that the model is consistent with the measurements. This approach is later extended for the situations where the signal fluctuates over time, e.g. non-steady-state sugar concentration. Andrews \textit{et al.} (2006) \cite{Andrews2006} propose that the cell adopts to such an environment by updating its beliefs in real-time. They model this decision-making mechanism as a sequential application of Bayesian inference, where the posterior probability that is inferred in the current step is used as a prior probability in the next step.\par
Bosher and Swain (2014) \cite{Bowsher2014} take a similar approach to explain cellular decision making, focusing on information theory. They argue that if the mutual information between the signal to be inferred and the output of a signal transduction mechanism is high, only then the cell could be able to perform a high-quality inference. For example, Dubuis \textit{et al.} (2013) \cite{dubuis2013}, in their study on embryonic development of fruit flies, show that the mutual information between gap gene expression and the position of the nucleus is the information needed for each cell to determine their position along the long axis of the embryo. \par
The cooperation of the agents is optimized through the exchange of information between the agents. In honeybee and ant colonies, individuals share information about potential nest sites with the colony through recruitment signals, which are accumulated to acquire an intelligent collective decision on the best nest site \cite{Franks2002}. In bacteria, individuals communicate by quorum sensing where they sense and produce hormone-like molecules called autoinducers. This process enables bacteria to observe the environment for the changes in the population or the presence of other bacteria and respond accordingly \cite{waters2005quorum}.\par
The influence of communication on cooperative behaviour in stochastic environments is an area of ongoing research. Paarporn \textit{et al.} (2018) \cite{Paarporn2018a} study the optimal behaviour in a two-player cooperative game in a stochastic environment with and without information sharing in the presence of noise in communication channels. They find that the majority logic strategy is optimal when the agents have intermediate reliable information from the environment and the information sharing between them is highly reliable.

\section{Motivation}
Inspired by the examples of multi-agent systems in nature, we are interested in getting insights about communication in a system where individuals rely on the exchange of information to achieve optimal behaviour. To this end, we consider a communication problem between individuals which evolve continuously in time. The messages sent in the system are translated by a language model. The translations are observed by an acting agent which then performs some task in coordination with others. We assume that the behaviour of the agent has been shaped by evolution (close) to optimality and that the optimal behaviour is known. The goal of this work is to infer the language model from demonstrations. \par
As a motivating example to this work, consider the state of a cell expressed in terms of its gene activation state. Based on this gene state, the cell may emit some message. Another cell, which we referred to as agent, receives a translation of the messages emitted by its neighbouring cells. Given some task and the corresponding optimal policy, the agent cell gauges its gene state in coordination with its neighbours.\par
We consider three individuals participating in the communication, two of which share information, and an agent which processes these messages. We model the individuals as components of a continuous-time Bayesian network (CTBN). The states of the components are emitted as messages. The agent does not have access to these messages directly but observes a translation of them, which poses a partially observable Markov decision process (POMDP). The agent taking an action corresponds to changing its dynamics. Our dataset consists of discrete-valued state trajectories of the components in CTBN, from which we infer the language model, that leads to the translated messages. As it is modelled within the POMDP framework, we refer to the language model as observation model in the remainder of the thesis.\par
The observation model can be interpreted in several ways which provide different scenarios to the problem that we consider. It can be interpreted as the communication protocol, where the individuals choose to share limited information with the population based on their states. The model can also be considered as a noisy communication channel or a summary/aggregation imposed on the messages by the environment. In the following, we focus on the latter.
\section{Contributions}
The main contributions of this thesis are as follows:
\begin{itemize}
	\item Formulation of the information transfer with a continuous-time graph-based approach
	\item Implementation of the system employing CTBN and POMDP frameworks and synthetic data generation
	\item Approximation of the belief state through marginalized CTBN
	\item Inference of the observation model from continuous-time demonstrations
	\item Evaluation of the performance on the synthetic data
\end{itemize}
\section{Structure of the Thesis}
The remainder of the thesis is structured as follows:

\textbf{\cref{chap:2}} presents the theoretical background of this work. It reviews the details of continuous-time Bayesian networks and partially observable Markov decision processes and introduces the sampling algorithms used in this work.

\textbf{\cref{chap:3}} is dedicated to the details of the problem. It explains how the frameworks are utilised, and presents the algorithms used for data generation and inference.

\textbf{\cref{chap:4}} presents the experimental results of the simulation and the inference.

\textbf{\cref{chap:5}} discusses the results, highlights the conclusions and reviews the limitations. Moreover, it gives suggestions for future directions and extentions to this work.