\chapter{Introduction}
%survıval requıres cooperatıon or ıs verz benefıtıal, cell stuff example, tuff lıfe. cooperatıon comes from communcatıon transmıttıng/exchage ınformatıon. the real world ıs noısy and unprecıse -> stochastıc, partly observable and stuff.
%A paragraph goes here where somehow stochasticity, cooperation and communication is mentioned together. maybe give some examples as problems or as areas of research. Where was such problems considered before? something like that.
In multi-agent systems, the well-being of the population strongly relies on the cooperation between the agents. Such systems can be found in nature from cellular level to swarms \cite{Perkins2009a, Tan2013}. They consist of relatively simple individuals and as limited abilities as these individuals might have, they give rise to a complex behaviour of the population through cooperation. These complex behaviors might be crucial for the survival of the population, e.g. running away from predators, partitioning of vital substances. \\
The cooperation of the agents is optimized through the exchange of information between the agents. The individual interests might conflict, as in many cases it would. However, the blabla. \\
Cells exist in stochastic environments. In order to maintain life, they are required to process noisy and fluctuating information coming from environment and response accordingly. In addition to the extracellular environment, the internal dynamics of cells are also found to be stochastic, such as their gene expressions \cite{Shahrezaei2008}. \\
Perkins \textit{et al.} \cite{Perkins2009a} argue that, as the extracellular environment is a stochastic process, the intracellular processing of these stochastic signals and choosing an appropriate response can only be probabilistic and consists of three main steps. First step is to infer from the noisy signals and make a prediction of the current or future state. Second is to make a proper choice of action considering the advantages and disadvantages, given the predictions. Final step is to take this action in a way that will contribute to the survival of the cell population. \\
Many studies took probabilistic approaches to explain the behaviour of cellular networks quantitatively and statistical inference is presented as a possible framework to explain the mechanisms that a cell may use to interpret the state of the environment from noisy signals. Libby \textit{et al} \cite{Libby2007} used Bayesian inference approach to model the gene expression of a bacterium in an environment with high and low level of sugar, and showed this model is consistent with the measurements. This approach is later extended for the situations where the signal fluctuates over time, e.g. non steady-state sugar concentration. Andrews \textit{et al} \cite{Andrews2016} proposed that the cell adopts to such environment by updating its beliefs in real-time. They modelled this decision-making mechanism as a sequential application of Bayesian inference, where the posterior probability that is inferred in the current step is used as a prior probability in the next step.\\
Bosher \textit{et al.} \cite{Bowsher2014} took a similar approach to explain cellular decision making, focusing on information theory. They argued that mutual information between the signal and the response is shown to be a suitable measure to quantify the cell's ability to infer. It is argued that if the mutual information between the signal to be inferred and the output of a signal transduction mechanism is high, only then the cell could be able to perform a high quality inference. For example, Bialek \textit{et al}, in their study of development of fruit fly, showed that the mutual information between gap gene expression and the position of nucleus is the information needed for each nucleus to determine their position in a cell. \\
The importance of information sharing in multi-agent systems are shown by blabla.
This paper tries to explain the effect of information sharing on the cooperative optimal strategy in a stochastically fluctuating environment. A coordination game of two players is presented in a two-state environment, where the players should coordinate on the two possible actions accordingly. The optimal strategy problem in this game is solved in two different communication scenarios. In the first game scenario $ G_{p} $, the players only get individual signals from the environment, and do not communicate with each other. The communication with the environment is modelled as binary noisy channel with probability $ p $ of getting the true signal. In the second scenario $ G_{pq} $, the players also share information about the cues they get from the environment, through another binary noisy channel with probability $ q $ of sending it correctly. With game-theoretic approach, the average long-term fitness of the players is defined as the fraction of time that the players cooperatively choose the correct action weighted by the game payoffs. The strategies maximizing the long-term fitness are considered optimal, and used to investigate the influence of information sharing on this cooperative game.

\section{Motivation}




\section{Structure of the Thesis}
The remainder of the thesis is structured as follows.

\textbf{\cref{chap:2}} presents the theoretical background of this work. It reviews the details of continuous-time Bayesian networks and partially observable Markov decision processes, and introduces the sampling algorithms used in this work.

\textbf{\cref{chap:3}} is dedicated to the details of the problem. It explains how the frameworks are utilised, and presents the algorithms used for data generation and inference.

\textbf{\cref{chap:4}} presents the experimental results of the simulation and the inference.

\textbf{\cref{chap:5}} discusses the results, highlights the conclusions and reviews the limitations.

\textbf{\cref{chap:6}} gives suggestions for future directions and extentions to the research.