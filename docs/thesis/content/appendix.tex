%\externaldocument[-f]{c2_foundations}
\appendix

\chapter{Amalgamation Operation}
\label{ap:amalgamation}
A CTBN with multiple variables can be represented with a single CIM. This is done by amalgamation operation. Amalgamation defines a combining operation over multiple CIMs and produces a single CIM for the entire system. \cite{Nodelman1995}

\section{Amalgamation of Independent Processes}
Consider a CTBN with graph $ \mathcal{G} = \left\lbrace \mathcal{V}, \mathcal{E} \right\rbrace $ over two variables such that $ \mathcal{V} = \left\lbrace X_1, X_2\right\rbrace $. Assume variables $ X_1 $ and $ X_2 $ with intensity matrices $ \textbf{Q}_1 $ and $ \textbf{Q}_2 $, are both parent nodes, i.e. $ \mathcal{E} = \emptyset $ and $ Par_{\mathcal{G}}(X_1) = Par_{\mathcal{G}}(X_1) = \emptyset $. This CTBN can be identified as a subsytem of the CTBN model described in \cref{sec:exp_ctbn_model}. \\
Analogous to \autoref{eq:Markov_trans_func}, Markov transition function for the joint process can be derived as
\begin{align}
	\operatorname{Pr}(X_P(t+h) = x_p^\prime\mid X_P(t) = x_p)
	&=  \operatorname{Pr}(X_1(t+h) = x_1^\prime, X_2(t+h) = x_2 \mid X_1(t) = x_1, X_2(t) = x_2)\nonumber \\
	&= \operatorname{Pr}(X_1(t+h) = x_1^\prime \mid X_1(t) = x_1, X_2(t) = x_2) \nonumber\\
	& \quad \quad \quad \operatorname{Pr}( X_2(t+h) = x_2 \mid X_1(t) = x_1, X_2(t+h) = x_2) \nonumber\\
	& = (\delta_{x_1^\prime, x_1} + hq^1_{x_1, x_1^\prime} + o(h))(1 + hq^2_{x_2, x_2} + o(h))\nonumber\\
	& = \delta_{x_1^\prime, x_1} + hq^1_{x_1, x_1^\prime} + h\delta_{x_1^\prime, x_1}q^2_{x_2, x_2} + o(h)
	\label{eq:amalg}
\end{align}
where $ x_1, x_1^\prime \in \rchi_1 $, $ x_2, x_2^\prime \in \rchi_2 $, $ x_p = (x_1, x_2), x_p^\prime = (x_1^\prime, x_2) \in \rchi_P $.\\
Suppose the intensity matrices of $ X_1 $ and $ X_2 $ are in the form
\begin{equation}
\textbf{Q}_i = 
\begin{bmatrix}
-q^{i}_{0} & q^{i}_{0} \\
q^{i}_{1} & -q^{i}_{1}
\end{bmatrix} \quad \text{for } i \in \left\lbrace 1,2\right\rbrace 
\end{equation}
Then the intensity matrix for the joint process $ X_P $ with factorising state space $ \rchi_P = \rchi_1 \times \rchi_2 $ can be written as
\begin{equation}
\textbf{Q}_P = 
\begin{bmatrix}
-q^{2}_{0}-q^{1}_{0} & q^{2}_{0} & q^{1}_{0} & 0 \\
q^{2}_{1} & -q^{2}_{1}-q^{1}_{0} & 0 & q^{1}_{0} \\
q^{1}_{1} & 0 & -q^{1}_{1}-q^{2}_{0} & q^{2}_{0} \\
0 & q^{1}_{1} & q^{2}_{1} & -q^{1}_{1}-q^{2}_{1}
\end{bmatrix} \quad \text{for } i \in \left\lbrace 1,2\right\rbrace 
\label{eq:amalgamated_q}
\end{equation}
As it can be observed from \autoref{eq:amalgamated_q}, the transition intensities which corresponds to state transition in both variables, i.e. anti-diagonal entries, are zero, due to the one of the assumptions in CTBN framework that only one variable can transition at a time, as given in \cref{sec:ctbn_intro}.
%x1 = Q1[0][1]
%x2 = Q1[1][0]
%y1 = Q2[0][1]
%y2 = Q2[1][0]
%[[0, y1, x1, .0],
%[y2, 0, .0, x1],
%[x2, .0, 0, y1],
%[.0, x2, y2, 0]])
\chapter{Marginalized Likelihood Function for Homogenous Continuous Time Markov Processes}
\label{ap:marg_llh_ctmp}

Let $ X $ be a homogenous CTMP. For convenience, it is assumed to be binary-valued, $ \rchi = \left\lbrace x_{0}, x_{1} \right\rbrace $. The transition intensity matrix can be written in the following form:
\begin{equation}
\textbf{Q} = 
\begin{bmatrix}
-q_{0} & q_{0} \\
q_{1} & -q_{1}
\end{bmatrix}
\end{equation}
where the transition intensities $ q_{0} $ and $ q_{1} $ are gamma-distributed with parameters $ \alpha_{0}$, $ \beta_{0} $ and $ \alpha_{1} $, $ \beta_{1} $, respectively. The marginal likelihood of a sample trajectory $ X^{[0,T]} $ can be written as follows:
\begin{align}
P(X^{[0, T]}) & = \int  P(X^{[0, T]}\mid Q)P(Q) dQ \nonumber\\ 
& = \int_{0}^{\infty} = \prod_{j \neq i}  \exp(-q_{i,j}T[x_{i}])\ q_{i,j}^{M[x_{i},x_{j}]} \frac{\beta_{i,j}^{\alpha_{i,j}}{q_{i,j}^{\alpha_{i,j}-1}}\exp(-\beta_{i,j}q_{i,j})}{\Gamma(\alpha_{i,j})} \ dq_{i,j} \nonumber\\ 
& = \prod_{i\in{0,1}}\int_{0}^{\infty} q_{i}^{M[x_{i}]} \ \exp(-q_{i}T[x_{i}]) \  \frac{\beta_{i}^{\alpha_{i}} \ q_{i}^{\alpha_{i}-1}\ \exp(-\beta_{i}q_{i})}{\Gamma(\alpha_{i})} \ dq_{i} \nonumber\\ 
& = \prod_{i\in{0,1}} \frac{\beta_{i}^{\alpha_{i}}}{\Gamma(\alpha_{i})} \int_{0}^{\infty} q_{i}^{M[x_{i}] + \alpha_{i} -1} \ \exp(-q_{i}(T[x_{i}]+\beta_{i})) \ dq_{i} \label{eq:wolfram_line}\\ 
& = \prod_{i\in{0,1}} \frac{\beta_{i}^{\alpha_{i}}}{\Gamma(\alpha_{i})} \left( -(T[x_{i}]+\beta_{i})^{-M[x_{i}] - \alpha_{i}}\ \Gamma(M[x_{i}] + \alpha_{i}, \ q_{i}(T[x_{i}]+\beta_{i})) \right) \Big|_0^\infty  \nonumber\\ 
& = \prod_{i\in{0,1}} \frac{\beta_{i}^{\alpha_{i}}}{\Gamma(\alpha_{i})} \left( (T[x_{i}]+\beta_{i})^{-M[x_{i}] - \alpha_{i}}\ \Gamma(M[x_{i}] + \alpha_{i}) \right)
\label{eq:Marg_traj}
\end{align}
%where $ T[x_{i}] $, the amount of time spent in state x, $ M[x,x'] $ the number of transitions from state x to x' and  $ M[x] = \sum_{x\neq x'}M[x,x'] $.\\

In \autoref{eq:wolfram_line}, the integral is solved using computer algebra system WolframAlpha as follows:
\begin{align}
\int x^{a} \ \exp(-xb) \ dx = -b^{-a-1} \ \Gamma(a+1, \ bx) + C
\label{eq:integral}
\end{align}

