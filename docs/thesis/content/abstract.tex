\newpage \
\newpage

\begin{abstract}
Multi-agent systems in nature, such as a population of cells, cooperate through information sharing. This information can be incomplete and noisy. Inspired by such systems, we consider a communication between three individuals, which are evolving continuously in time. Two of them act independently, and emit messages containing information about their states, while the third one receives a translation of these messages. Based on these translated observations, the agent node form its belief over the state of other nodes. We modelled this system combining continuous-time Bayesian network (CTBN) and partially observable Markov decision (POMDP) process frameworks. The nodes evolve continously in time as components of a CTBN. Given that the true messages are unavailable to the agent, the interaction between the nodes is modelled as POMDP. The exact update of the belief state is computed by filtering, and these results are used as a baseline. The approximation of the belief state is obstained using marginalized particle filtering. The aim of this work is to infer the language model which leads to the translations.\\
%The belief state is updated utilising two methods. The first one is exact update, discussed in \cref{par:bs_exact}, and assumes that the transition intensities of the parents $ \textbf{Q}_1 $ and $ \textbf{Q}_2 $ are available both for the agent and for the classifier. However, due to the fact that this would not present a realistic system, particle filtering with marginalized CTBN is introduced for as state estimator. Here, both the agent and the classifier was able to perform the belief state update, given Gamma-priors of $ \textbf{Q}_1 $ and $ \textbf{Q}_2 $.\\
\end{abstract}

\newpage \
\newpage