\documentclass[]{article}
\usepackage[a4paper, total={6in, 9in}]{geometry}
\usepackage{amsmath}
\usepackage{algorithmic}
\usepackage{mathtools}
\usepackage[ruled, lined, longend]{algorithm2e}
\SetKwProg{Init}{Initialize}{}{}
%opening
\title{Likelihood model}
\author{Gizem Ekinci, Anam Tahir, Dominik Linzner}

\begin{document}
	
	\maketitle
	Let \textit{D} be a sample of trajectories in the dataset, such that $ \textit{D} = \left\langle X_{1}^{[0, T]}, X_{2}^{[0, T]}, X_{3}^{[0, T]} \right\rangle  $, and the set of parameters to the system $  \theta = \left\langle Q_{1}, Q_{2}, \pi, \Phi \right\rangle  $, where $ \Phi $ is observation model, $ \pi $ is optimal stochastic policy, $ Q_{1} $ and $ Q_{2} $ are the transition intensity matrices of $ X_{1} $ and $ X_{2}$, respectively. Then likelihood of the sample trajectory \textit{D} can be written as:
	\begin{align}
		P(\textit{D} \mid \theta ) & = P(X_{1}^{[0, T]}, X_{2}^{[0, T]}, X_{3}^{[0, T]} \mid Q_{1}, Q_{2}, \pi, \Phi) \\ & = P(X_{3}^{[0, T]} \mid X_{1}^{[0, T]}, X_{2}^{[0, T]}, Q_{1}, Q_{2}, \pi, \Phi) \ P(X_{1}^{[0, T]}\mid Q_{1}) \ P(X_{2}^{[0, T]}\mid Q_{2}) \\ & = P(X_{3}^{[0, T]} \mid X_{1}^{[0, T]}, X_{2}^{[0, T]}, \pi, \Phi) \ P(X_{1}^{[0, T]}\mid Q_{1}) \ P(X_{2}^{[0, T]}\mid Q_{2}) \\ & = P(X_{3}^{[0, T]}\mid Q_{3}^{[0, T]}) \ P(X_{1}^{[0, T]}\mid Q_{1}) \ P(X_{2}^{[0, T]}\mid Q_{2}) 
	\end{align}
	 where $ Q_{3}^{[0, T]} $ is the trajectory of transition intensity matrix of $X_{3}$ and is a deterministic function of $ X_{1}^{[0, T]}, X_{2}^{[0, T]}, \pi $ and $ \Phi $. \\
	 
	Marginalizing the likelihood over $ Q_{1} $ and $ Q_{2} $:
	\begin{align}
	P(\textit{D} \mid \pi, \Phi ) & = 	\int \int P(\textit{D} \mid \theta ) \ P(Q_{1}) \ P(Q_{2}) \ dQ_{1}dQ_{2} \\ & = \int \int P(X_{3}^{[0, T]}\mid Q_{3}^{[0, T]}) \ P(X_{1}^{[0, T]}\mid Q_{1}) \ P(X_{2}^{[0, T]}\mid Q_{2}) \ P(Q_{1}) \ P(Q_{2})\ dQ_{1}dQ_{2} \\ & = P(X_{3}^{[0, T]}\mid Q_{3}^{[0, T]}) \int  P(X_{1}^{[0, T]}\mid Q_{1}) \ P(Q_{1}) \ dQ_{1} \int P(X_{2}^{[0, T]}\mid Q_{2})\ P(Q_{2})\ dQ_{2}
	\label{eq:Marg_llh}
	\end{align}
	
	$ X_{1} $ and $ X_{2} $ are independent homogenous Markov processes, with state space $ Val(X_{1, 2}) = \left\lbrace 0, 1 \right\rbrace $. The transition intensity matrices $ Q_{1} $ and $ Q_{2} $ can be written in the following form for convenience,
	\begin{center}
		\begin{math}
		\begin{pmatrix}
		-q_{0} & q_{0} \\
		q_{1} & -q_{1}
		\end{pmatrix}
		\end{math}\\
	\end{center}
	where the transition intensities $ q_{0} $ and $ q_{1} $ are gamma-distributed with parameters $ \alpha_{0}$, $ \beta_{0} $ and $ \alpha_{1} $, $ \beta_{1} $, respectively. The marginal likelihood of a sample trajectory from binary-valued homogenous Markov process X with transition intensity matrix Q can be written as follows:
	
	\begin{align}
	P(X^{[0, T]}) & = \int  P(X^{[0, T]}\mid Q)P(Q) dQ \\ & = \int_{0}^{\infty} \left( \prod_{x} \exp(-q_{x}T_{x}) \right) \left( \prod_{x'} q_{xx'}^{M[x, x']}\right) \frac{\beta_{xx'}^{\alpha_{xx'}}{q_{xx'}^{\alpha_{xx'}-1}}\exp(-\beta_{xx'}q_{xx'})}{\Gamma(\alpha_{xx'})} \ dq_{xx'} \\ & = \prod_{x\in{0,1}}\int_{0}^{\infty} q_{x}^{M_{x}} \ \exp(-q_{x}T_{x}) \  \frac{\beta_{x}^{\alpha_{x}} \ q_{x}^{\alpha_{x}-1}\ \exp(-\beta_{x}q_{x})}{\Gamma(\alpha_{x})} \ dq_{x} \\ & = \prod_{x\in{0,1}} \frac{\beta_{x}^{\alpha_{x}}}{\Gamma(\alpha_{x})} \int_{0}^{\infty} q_{x}^{M_{x} + \alpha_{x} -1} \ \exp(-q_{x}(T_{x}+\beta_{x})) \ dq_{x} \\ & = \prod_{x\in{0,1}} \frac{\beta_{x}^{\alpha_{x}}}{\Gamma(\alpha_{x})} \left( -(T_{x}+\beta_{x})^{M_{x} + \alpha_{x}}\ \Gamma(M_{x} + \alpha_{x}, \ q_{x}(T_{x}+\beta_{x})) \right) \Big|_0^\infty  \\ & = \prod_{x\in{0,1}} \frac{\beta_{x}^{\alpha_{x}}}{\Gamma(\alpha_{x})} \left( (T_{x}+\beta_{x})^{M_{x} + \alpha_{x}}\ \Gamma(M_{x} + \alpha_{x}) \right)
	\label{eq:Marg_traj}
	\end{align}
	
	where $ T_{x} $, the amount of time spent in state x, $ M[x,x'] $ the number of transitions from state x to x' and  $ M[x] = \sum_{x\neq x'}M[x,x'] $.\\
	
	From Eq.11, the integral is solved using computer algebra system WolframAlpha as follows:
	\begin{align}
	\int x^{a} \ \exp(-xb) \ dx = -b^{-a-1} \ \Gamma(a+1, \ bx) + C
	\label{eq:integral}
	\end{align}
	
	Plugging Eq.\ref{eq:Marg_traj} in Eq.\ref{eq:Marg_llh} for both $ X_{1} $ and $ X_{2} $:
	\begin{align}
	\begin{split}
		P(\textit{D} \mid \pi, \Phi ) = P(X_{3}^{[0, T]}\mid Q_{3}^{[0, T]}) \prod_{x_{1}\in{0,1}} \frac{\beta_{x_{1}}^{\alpha_{x_{1}}}}{\Gamma(\alpha_{x_{1}})} \ (T_{x_{1}}+\beta_{x_{1}})^{M_{x_{1}} + \alpha_{x_{1}}}\ \Gamma(M_{x_{1}} + \alpha_{x_{1}})  \\  \prod_{x_{2}\in{0,1}} \frac{\beta_{x_{2}}^{\alpha_{x_{2}}}}{\Gamma(\alpha_{x_{2}})} \ (T_{x_{2}}+\beta_{x_{2}})^{M_{x_{2}} + \alpha_{x_{2}}}\ \Gamma(M_{x_{2}} + \alpha_{x_{2}})
	\label{eq:Marg_llh_final}
	\end{split}
	\end{align}
	
\end{document}
